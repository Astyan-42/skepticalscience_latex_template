\documentclass[a4paper, oneside, twocolumn, 11pt]{ssjtwo}
\usepackage{lipsum}


\begin{document}


\title{Title of the publication} 
\author[1]{Author A}
\author[1]{Author B\thanks{B.B@university.edu}}
\author[1]{Author C}
\author[2]{Author D\thanks{D.D@university.edu}}
\author[2]{Author E}
\affil[1]{Department of Computer Science, \LaTeX\ University}
\affil[2]{Department of Mechanical Engineering, \LaTeX\ University}

% \contributor{Test Contributor}

%----------------------------------------------------------------------------------------

\maketitle 

\begin{article}

\begin{abstract}
\lipsum[1]
\end{abstract}


%\keywords{Keyword1 | Keyword2 | Keyword3} 
%\abbreviations{SAM, self-assembled monolayer; OTS, octadecyltrichlorosilane}

\section{Introduction}

\lipsum[1] \cite{CLAcha1}.

\begin{equation}
\mfrac{D \theta}{Dt}=\mfrac{ \theta}{ t} + u\cdot \nabla
\theta=0 \label{qg1}
\end{equation}

Referencing equation \eqref{qg1}. \lipsum[1]

\section{Results}

\lipsum[1]. Referencing Table \ref{sampletable}. Referencing Figure \ref{placeholder}.

\subsection{Simulations}

\subsubsection{Simulation 1}

\lipsum[1]

\subsubsection{Simulation 2}

\lipsum[1]

\subsection{Real Data}

\lipsum[1]

\section{Discussion}

\lipsum[1]

\lipsum[1]

\lipsum[1]

\begin{materials}

\lipsum[1]

\begin{definition}
A bounded function $\theta$ is a weak solution of QG if for any
$\phi\,\epsilon\, C_0^{\infty}
(\mathbb{Z}\times\mathbb{R}
\times[0,\epsilon ])$ we have
\begin{eqnarray}
&&  \int_{\mathbb{R}^+\mathbb{R}}
 \theta(x,y,t)\, \phi
\,(x,y,t) dy dx dt+\nonumber\\
  & +&\int_{\mathbb{R}^+\mathbb{R}}
 \theta\,(x,y,t) u(x,y,t)\cdot\nabla\phi\,(x,y,t)
dydxdt = 0 \label{weaksol} \end{eqnarray}
where $u$ is determined previously.
\end{definition}

\lipsum[1]

\begin{theorem}
If the active scalar $\theta$ satisfies
the equation \eqref{weaksol}, then $\varphi$ satisfies the equation
with $|Error|\leq C\, \delta | log\delta| $ where $C$ depends only
on $\|\theta\|_{L^{\infty}}$ and $\|
\nabla\varphi\|_{L^{\infty}}$.
\end{theorem}

\lipsum[1]
\end{materials}


\begin{acknowledgments}
\lipsum[1]
\end{acknowledgments}

\end{article}


\begin{figure}[h]
\centerline{\includegraphics[width=0.4\linewidth]{placeholder.jpg}}
\caption{Figure caption}\label{placeholder}
\end{figure}

\begin{table}[h]
\caption{Table caption}\label{sampletable}
\begin{tabular}{l l l}
\hline
\textbf{Treatments} & \textbf{Response 1} & \textbf{Response 2}\\
\hline
Treatment 1 & 0.0003262 & 0.562 \\
Treatment 2 & 0.0015681 & 0.910 \\
Treatment 3 & 0.0009271 & 0.296 \\
\hline
\end{tabular}
\end{table}


%\begin{figure*}
%\caption{Almost Sharp Front}\label{afoto}
%\end{figure*}

%\begin{table*}
%\caption{Repeat length of longer allele by age of onset class}
%\begin{tabular}{ccc}
%\end{tabular}
%\end{table*}

\end{document}